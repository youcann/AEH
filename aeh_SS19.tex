\documentclass[12pt,a4paper]{scrartcl}
\usepackage[utf8]{inputenc}
\usepackage[T1]{fontenc}
\usepackage[ngerman]{babel}
\usepackage[pdftex]{graphicx}

\usepackage{latexsym}
\usepackage{amsmath,amssymb,amsthm}

\setlength{\topmargin}{-15mm}
\newtheorem{Satz}{Satz}[section]
\newtheorem{Definition}[Satz]{Definition} 
\newtheorem{Lemma}[Satz]{Lemma}		   
\numberwithin{equation}{section} 


%%%%%%extra stuff%%%%%%%%%
\usepackage{tikz}
\usepackage{tikzscale}
\usepackage{pgfplots}
\usetikzlibrary{arrows.meta}
\usetikzlibrary{plotmarks}
\usetikzlibrary{matrix}
\pgfplotsset{compat=1.5.1}
\usepackage[list=false]{subcaption}
\usepackage[export]{adjustbox}
\usetikzlibrary{positioning}
\usetikzlibrary{positioning,fit,backgrounds,arrows,shapes,automata,petri,calc,bending}
\usepackage{tabularx}


%paragraphs
\newcommand{\properparagraph}[1]{\paragraph{#1}\mbox{}\\}
\usepackage[parfill]{parskip}

%floats
\usepackage{float}

%kluwe-specific specials
\newcommand{\beispiel}{\underline{Beispiel:~}}
\usepackage{accents}
\newcommand{\ubar}[1]{\underaccent{\bar}{#1}}
\newcommand\m[1]{\begin{bmatrix}#1\end{bmatrix}}

\begin{document}
  \pagestyle{empty}
  \begin{titlepage}
    \includegraphics[scale=0.45]{kit-logo.jpg} 
    \vspace*{2cm} 

 \begin{center} \large 
    Mitschrift
    \vspace*{2cm}

    {\huge AEH}
    \vspace*{2.5cm}

    Marvin Noll
    \vspace*{1.5cm}

    SS19
    \vspace*{4.5cm}


    %Betreuung: Name der Betreuerin / des Betreuers \\[1cm]
    %Fakultät für Mathematik \\[1cm]
		Karlsruher Institut für Technologie
  \end{center}
\end{titlepage}
  \tableofcontents
\newpage
  \pagestyle{headings}
  
  
  

%Vorlesung vom 29.04.19
\section*{Vorbemerkungen}
\begin{itemize}
	\item Ilias: MaxPlus2019
	\item Vorlesungsart: Tafel + Beiblätter
	\item Prüfung: mündlich, Termine über das ganze Jahr möglich
\end{itemize}

\section{Einleitung}
Grundaufgaben zur Steuerungs- und Regelungstechnik:
\begin{itemize}
	\item Prozessmodellierung
	\item Analyse (Simulation)
	\item Synthese von Regelungen und Steuerungen 
\end{itemize}
Grundlage: \underline{Systemtheorie}

\subsection{Systemklassifikation}
\begin{itemize}
\item Systemeinteilung nach den wesentlichen Prozessvorgängen möglich (\textit{BB. AEH 1-1})
\item Zuordnung Vorgänge - Prozesstypen (\textit{BB. AEH 1-2})
\item Vorraussetzung für sinnvolle Prozessbeschreibung
\end{itemize}

\subsubsection{Festlegung des Abstraktionsniveaus}
Beispiele:

\properparagraph{1) Aufzug}
%TODO Abb

\properparagraph{2) Tank}
(\textit{BB. AEH 1-3})

Abstraktion bedingt durch die AT-Aufgabe und/oder die Randbedingungen (z.B. Aktorik/Sensorik).

Also: Modellierung kombiniert Aufgaben(Ziele) und Prozesswissen(Physik, Randbedingungen)

\subsubsection{Wissensmodellierung und Verarbeitung}
(\textit{BB. AEH 1-4})

\subsection{Begriffsbestimmungen}
Erforderlich zur Modellformalisierung

zentrale Begriffe:
\begin{itemize}
	\item System 
	\item Zustand
	\item Ereignis 
	\smash{\raisebox{.5\dimexpr2\baselineskip+4\itemsep+2\parskip}{$\left.\rule{0pt}{.5\dimexpr3\baselineskip+3\itemsep+3\parskip}\right\}\text{Dynamisches System}$}}
\end{itemize}

(\textit{BB. AEH 1-5})

\properparagraph{Beispiel Tank (Dynamisches System)}

%TODO Abb.

\underline{Ereignisse}
\begin{itemize}
	\item \textbf{autonom}: (Größenänderung aufgrund interner Zustandsübergänge)
	\item \textbf{gesteuert}: (Größenänderung aufgrund externer Stellsignale)
\end{itemize}

\properparagraph{Beispiel Zulaufventil}

%TODO Abb.

\underline{In dieser Vorlesung}

Zwei besondere Ausprägungen dynamischer Systeme betrachtet:
\begin{enumerate}
	\item Ereignisdiskrete Systeme
	\smash{\raisebox{-.5\dimexpr1\baselineskip+4\itemsep+2\parskip}{$\left.\rule{0pt}{.5\dimexpr2\baselineskip+3\itemsep+3\parskip}\right\}\text{Dynamisches System}$}} 
	\item Hybride Systeme 
\end{enumerate}

\underline{Dynamisches System}

\begin{figure}[H]
	\centering
	\resizebox{!}{!}{\includegraphics[width=\textwidth]{img/2019_04_29_Abb5.tikz}}
\end{figure}

\properparagraph{Inhalte der Vorlesung}

\begin{itemize}
	\item Ereignisdiskrete Systeme: 
	\begin{itemize}
		\item Modelltypen und Prozessmodellierung
		\item Analyse und Simulation
		\item Spezifikation und Synthese von Steuerungen
	\end{itemize}
	\item Hybride Systeme
	\begin{itemize}
		\item Modellierung
		\item Simulation, Analyse, Steuerung
	\end{itemize} 
\end{itemize}


\subsection{Beispiel: gesteuerter Chargenprozess}
Prozessgeführte Ablaufsteuerung (Struktur siehe \textit{BB. AEH 1-6})

Prinzip: Einfüllen - Heizen - Entleeren einer Produktmenge(=Charge) [Füllhöhe $h$, Temperatur $\vartheta$] mittels der Stellgrößen [Ein- und Auslassventile $v_1$, $v_2$, Heizung $H$] durch eine diskrete Steuerung.


%Vorlesung vom 06.05.19
\section{Modelltypen und Beschreibungsformen ereignisdiskreter Systeme}

\underline{Ereignisdiskrete Systeme (DES)}

diskreter, abzählbarer Zustandsraum ohne natürliche Ordnung der Zustände.
\begin{itemize}
	\item kein Abstandsmaß für eine eventuelle differentielle/integrale Betrachtung
	\item Analyse und Synthese meist auf Suchalgorithmen basierend (Problem: kombinatorische Explosion)
\end{itemize}

\underline{Übersicht über Modelltypen und Beschreibungsformen}

(s. \textit{BB. AEH 2-1})

\properparagraph{Übergangsverhalten: beschreibt Zustandsübergänge}
%TODO Abb.

\properparagraph{Zeitverhalten: beschreibt Zeitabhängigkeit der Zustandsübergänge}
%TODO Abb.

\underline{hier:}

\begin{itemize}
	\item Automaten, Petri-Netze, NCE-Systeme
	\item Formale Sprachen, Max-Plus-Algebra
\end{itemize}

\subsection{Automaten und formale Sprachen}
\begin{itemize}
	\item zu Grundlagen der theoretischen Informatik gehörig
	\item älteste DEM-Modellierung (seit 1943)
\end{itemize}

\subsubsection{Automaten}
\underline{hier:} endliche Automaten betrachtet. (Def s. \textit{BB. AEH 2-2})

\properparagraph{Repräsentation}
\begin{itemize}
	\item Automatentafel (Transitionstafel)
	\smash{\raisebox{-.5\dimexpr1\baselineskip+4\itemsep+2\parskip}{$\left.\rule{0pt}{.5\dimexpr2\baselineskip+3\itemsep+3\parskip}\right\}\text{(s. \textit{BB AEH 2-2})}$}} 
	\item Automatengraph 
\end{itemize}

\underline{Beispiel: } Mealy-Automat (s. \textit{BB AEH 2-5})

\subsubsection{Formale Sprachen}
formales Beschreibungsmittel für DES

\properparagraph{Grundbegriffe}
(s. \textit{BB AEH-2-3})

\underline{Beispiel:}
\begin{equation}
	\underbrace{E}_\text{Alphabet} = \{\underbrace{e_1, e_2, e_3}_\text{Symbole}\}
\end{equation}

\begin{equation}
\text{Wörter}\left\{
\begin{aligned}
s_1 &= e_2 \\
s_2 &= e_1 e_2 \\
\end{aligned}
\right.
\end{equation}

\begin{equation}
	L_1(E) = \{s_1, s_2\} = \{e_1, e_1 e_3\} 
\end{equation}

\begin{equation}
	L_2(E) = \{ \text{Alle Wörter der Länge } n \}
\end{equation}

\begin{equation}
	L_1^{\ast}(E) = \{ \underbrace{\epsilon}_\text{leeres Wort}, e_2, e_1 e_3, e_2 e_2, e_2 e_1 e_3, e_1 e_3 e_1 e_3, \ldots \}
\end{equation}

\properparagraph{Zusammenhang mit den endlichen Automaten}
Reguläre Sprachen: (s. \textit{BB AEH 2-4})

es gilt: jede reguläre Sprache kann von einem endlichen Automaten repräsentiert werden.

\underline{Beispiel:} $L(E) = \{e_1^{\ast} e_2\}$

%TODO Abb.

analog: für jeden endlichen Automaten $A$ gibt es eine spezifische reguläre Sprache $L_A$, die er akzeptiert. (s. \textit{BB AEH 2-4})

\underline{Beispiel: } Mealy-Automat (auf \textit{BB AEH 2-5})
\begin{equation}
	L_A = \{ (e_3^{\ast}(e_1 (e_2 e_3)^{\ast} e_1 e_2)^{\ast})^{\ast} \}
\end{equation}
(s. \textit{BB AEH 2-5})

\underline{in Vorlesung: } Petri-Netze als Alternative DES-Modelle fokussiert.

\subsection{Petri-Netze}
\textsc{Karl-Adam Petri}: \textit{"Kommunikation mit Automaten"} (1926)

\underline{$\overbrace{\text{Petri-Netz}}^{\text{PN}}\text{-Darstellung}$}

\begin{itemize}
	\item \textbf{grafisch:} diputtdftdt gerichteter Graph (mit Zustandskennzeichnung)
	\item \textbf{algebraisch:} Netzmatrix $N$\footnote{Führt auf algebraische Gleichung für die Dynamik} (nur PN-Struktur) 
\end{itemize}

\subsubsection{Netztopologie}

\underline{PN-Elemente} (s. \textit{BB AEH 2-6})

\begin{tabularx}{0.9\linewidth}{p{2.1cm}p{0.1cm}X}
	Stellen & $\mathrel{\hat{=}}$ & Zustände \\
	Transitionen & $\mathrel{\hat{=}}$ & Ereignisse \\
	Kanten & $\mathrel{\hat{=}}$ & kausale Zusammenhänge \\
	Marken & $\mathrel{\hat{=}}$ & Kennzeichnung des aktuellen Zustands, Dynamik durch Netzparameter festgelegt \\
\end{tabularx}

\underline{Beispiel:}


\begin{subequations}
	\begin{align}
	S &= \{ s_1, s_2 \} \\
	T &= \{ t_1, t_2 \} \\
	F &= F_{pre} \cup F_{post} = \{ (s_1, t_1), (s_2,t_2) \} \cup \{(t_1, s_2), (t_2, s_1) \} \\
	K&:\; K(s_1)=1, \qquad K(s_2)=2 \\
	W&:\; W(s_1,t_1)=1, \qquad W(t_1,s_2)=2, \qquad W(s_2,t_2)=2, \qquad W(t_2,s_1)=1 \\
	M_0&:\; M_0(s_1)=1, \qquad M_0(s_2)=0
	\end{align}
\end{subequations}


für Dynamik (siehe Abschnitt~\ref{sec:netzdynamik}) Kennzeichnung von bestimmten PN-Bereichen sinnvoll (s. \textit{BB AEH 2-7})

\underline{Beispiel: }
%TODO Abb.

%Vorlesung vom 13.05.19

\subsubsection{Netzdynamik} \label{sec:netzdynamik}
$\hat{=}$ Markenfluss im PN, bewirkt durch \textit{Schaltvorgänge} aktivierter Transitionen (s. \textit{BB AEH 2-8})

\properparagraph{Beispiel}
%TODO Abbn.

\properparagraph{Konfliktlösung (mittels Zusatzwissen)}
\begin{itemize}
	\item zusätzliche bool'sche Bedingung an den Transitionen
	\item Kantenzeitbewertung (siehe später)
	\item alternierendes Schalten
	\item Zufallsauswahl
\end{itemize}

\properparagraph{Demo: DESSKA}
%TODO Abb.

\subsubsection{Netzklassen}
\underline{Netzklassen:} Netz mit bestimmten (beschränkten) Eigenschaften zur Abbildung der jeweils charakteristischen dynamischen Abläufe (Beschränkung vorteilhaft für die Analyse/Synthese).

Klassifizierung nach:
\begin{itemize}
	\item Struktureigenschaften
	\item Stellen-/Kanteneigenschaften 
	\smash{\raisebox{.5\dimexpr1\baselineskip+4\itemsep+2\parskip}{$\left.\rule{0pt}{.5\dimexpr2\baselineskip+3\itemsep+3\parskip}\right\}\text{(s. \textit{BB AEH 2-9 und AEH 2-10})}$}} 
\end{itemize}

\properparagraph{Erläuterungen Struktureigenschaften}
\begin{itemize}
	\item \textbf{\underline{Reines PN}}
	
	Schleifen verboten
	%TODO Abb.
	
	\item \textbf{\underline{Zustandsmaschine (ZM)}}
	
	\underline{Beispiel:}
	%TODO Abb.
	
	Kennzeichen:
	\begin{itemize}
		\item Abbildung alternativer Abläufe
		\item Markenzahl in ZM konstant
		\item Markenanzahl:
		\begin{itemize}
			\item mehrere Marken: Nebenläufigkeit darstellbar (aber: $\forall s_i: K(s_i)>1$ erforderlich)
			\item genau eine Marke: ZM $\hat{=}$ Automatengraph
		\end{itemize}
	\end{itemize}

	\item \underline{Synchronisationsgraph (SG)} (s. \textit{BB AEH 2-10})
	
	\underline{Beispiel:}
	%TODO Abb.
	
	Kennzeichen:
	\begin{itemize}
		\item Modellierung von Nebenläufigkeit
		\item keine Alternativen abbildbar (dadurch konfliktfrei)
		\item Markenzahl nicht konstant
	\end{itemize}

	\item \underline{Free-Choice-Netze (FC-Netz)} (s. \textit{BB AEH 2-10})
	
	Kennzeichen:
	\begin{itemize}
		\item Kombination von ZM- und SG-Strukturen \newline ($\Rightarrow$ ZM $\subset$ FC-Netz, SG $\subset$ FC-Netz)
		\item bei Alternativen ist nur der Ausgangszustand (Konfliktstelle) betroffen, kein Einfluss anderer Zustände (Stellen)
	\end{itemize}
\end{itemize}

\properparagraph{Anmerkungen}
\begin{enumerate}
	\item Erweiterung des FC-Netz: \underline{Extended Free-Choice-Netz (EFC-Netz)}
	
	gegenseitig bedingte Alternativen zulässig
	%TODO Abb.
	
	\item allgemeinstes PN: \underline{Stellen/Transitionen-Netz (S/T-Netz)}
	
	Alle Struktur- und Knoten/Kanteneigenschaften erlaubt (demzufolge alle übrigen Netzklassen).
\end{enumerate}

\properparagraph{Beispiele}
%TODO Abb.

%VL am 20.05.19

\subsubsection{Algebraische Netzbeschreibung}
Basis für die PN-Darstellung in Rechner und die PN-Analyse(s. Kapitel 4).

\underline{Begriffe:} (s. BB AEH2-11, AEH2-12)

\beispiel

\begin{figure}[H]
	\centering
	\includegraphics[width=5cm]{img/2019_05_20/abb1.tikz}
\end{figure}

\begin{subequations}
	\begin{align}
	\ubar{K}   &= \begin{bmatrix}10\\2\end{bmatrix} \\
	\ubar{t}_1 &= \begin{bmatrix}3\\-1\end{bmatrix}\\
	\ubar{t}_2 &= \begin{bmatrix}-1\\2\end{bmatrix}\\
	\ubar{N}   &= \begin{bmatrix}t_1 & t_2\end{bmatrix} = \begin{bmatrix}3 & -1 \\ -1 & 2\end{bmatrix}\\
	\ubar{m}_0 &= \begin{bmatrix}1 \\ 0\end{bmatrix}
	\end{align}
\end{subequations}

Schaltvoraussetzung:
\begin{subequations}
	\begin{align}
	t_1 &: 0 \le \underbrace{\m{1\\0}+\m{3\\-1}}_{\m{4\\-1}} \le \m{10\\2}\\[2ex]
	t_2 &: 0 \le \underbrace{\m{1\\0}+\m{1\\2}}_{\m{0\\2}} \le \m{10\\2}
	\end{align}
\end{subequations}

Schaltvektor: $t_2$ aktiviert $\Rightarrow \ubar{v}(1)=\m{0\\1}$

Folgemarkierung: $\ubar{m}(1)=\ubar{m}(0) + \ubar{N} \ubar{v}(1) = \m{1\\0}+\m{3&-1\\-1&2} \m{0\\1} = \m{0\\2}$

(weitere Schaltvorgänge analog)

\subsubsection{Darstellung von Nebenläufigkeit}
Untersuchung der Modellierungstransparenz bei Nebenläufigkeiten: PN $\leftrightarrow$ Automat.

\beispiel Zwei Prozesse

\begin{tabular}{l|l|l}
	 & Zustände & Ereignisse \\
	 \hline
	 $P_1$ & $z_1,z_2,z_3$ & $e_1,e_2,e_3$ \\
	 \hline
	 $P_2$ & $z_4,z_5,z_6$ & $e_4,e_5,e_6$ 
\end{tabular}

Nebenläufigkeiten: 
\begin{itemize}
	\item asynchron: Ereignisse treten unabhängig voneinander auf
	\item hier: $e_3$ und $e_6$ treten gleichzeitig auf
\end{itemize}

Vergleich: lokaler(getrennter) und globaler Automatengraph mit PNen (s. BB AEH 2-13)

Fazit: PN aufgrund der verteilten Zustandsinformation (Markierung) und den transparenten Strukturen (bei synchronisierten Nebenläufigkeiten) kompakter und übersichtlicher.

\subsubsection{Zeitbewertung}
Bei PN erforderlich, falls neben kausalen Ablauf auch zeitlicher Zusammenhang relevant. 

Prinzip: (s. BB AEH 2-14)\\
(Sonderfall: Ereignisse nicht zeitbewertet: $\tau_R=0$, $\tau_L=\infty$)

\beispiel Kleben von Teilen

\begin{figure}[H]
	\centering
	\includegraphics[]{img/2019_05_20/abb3.tikz}
\end{figure}

Weitere Beispiele (s. BB AEH 2-15)

\subsubsection{Test- und Inhibitorkanten}
Spezielle Kanten zur Einbringung zusätzlicher Bedingungen in die Dynamik ohne Markenfluss beim Schaltvorgang (s. BB AEH 2-16)

\beispiel Synchronisation ansonsten unabhängiger Prozesse
\begin{figure}[H]
	\centering
	\includegraphics[]{img/2019_05_20/abb4.tikz}
\end{figure}

$t_4$ immer erst nach dem Schalten von $t_3$ und $t_2$ aktiviert.

Vorteil: Übersichtliche Darstellung\\
Nachteil: Analyse erschwert

\subsection{Netz - Condition/Event - Systeme}
\underline{PN-Schwächen:}



\end{document}

