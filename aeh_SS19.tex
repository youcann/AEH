\documentclass[12pt,a4paper]{scrartcl}
\usepackage[utf8]{inputenc}
\usepackage[T1]{fontenc}
\usepackage[ngerman]{babel}

\usepackage[pdftex]{graphicx}
\usepackage{latexsym}
\usepackage{amsmath,amssymb,amsthm}

\setlength{\topmargin}{-15mm}
\newtheorem{Satz}{Satz}[section]
\newtheorem{Definition}[Satz]{Definition} 
\newtheorem{Lemma}[Satz]{Lemma}		   
\numberwithin{equation}{section} 


%%%%%%extra stuff%%%%%%%%%
\usepackage{tikz}
\usepackage{pgfplots}
\usetikzlibrary{arrows.meta}
\usetikzlibrary{plotmarks}
\usetikzlibrary{matrix}
\pgfplotsset{compat=1.5.1}
\usepackage[list=false]{subcaption}
\usepackage[export]{adjustbox}
\usetikzlibrary{positioning}
\usepackage{tikzscale}

%paragraphs
\newcommand{\properparagraph}[1]{\paragraph{#1}\mbox{}\\}
\usepackage[parfill]{parskip}

%floats
\usepackage{float}

\begin{document}
  \pagestyle{empty}
  \begin{titlepage}
    \includegraphics[scale=0.45]{kit-logo.jpg} 
    \vspace*{2cm} 

 \begin{center} \large 
    Mitschrift
    \vspace*{2cm}

    {\huge AEH}
    \vspace*{2.5cm}

    Marvin Noll
    \vspace*{1.5cm}

    SS19
    \vspace*{4.5cm}


    %Betreuung: Name der Betreuerin / des Betreuers \\[1cm]
    %Fakultät für Mathematik \\[1cm]
		Karlsruher Institut für Technologie
  \end{center}
\end{titlepage}
  \tableofcontents
\newpage
  \pagestyle{headings}

%Vorlesung vom 29.04.19
\section*{Vorbemerkungen}
\begin{itemize}
	\item Ilias: MaxPlus2019
	\item Vorlesungsart: Tafel + Beiblätter
	\item Prüfung: mündlich, Termine über das ganze Jahr möglich
\end{itemize}

\section{Einleitung}
Grundaufgaben zur Steuerungs- und Regelungstechnik:
\begin{itemize}
	\item Prozessmodellierung
	\item Analyse (Simulation)
	\item Synthese von Regelungen und Steuerungen 
\end{itemize}
Grundlage: \underline{Systemtheorie}

\subsection{Systemklassifikation}
\begin{itemize}
\item Systemeinteilung nach den wesentlichen Prozessvorgängen möglich (\textit{BB. AEH 1-1})
\item Zuordnung Vorgänge - Prozesstypen (\textit{BB. AEH 1-2})
\item Vorraussetzung für sinnvolle Prozessbeschreibung
\end{itemize}

\subsubsection{Festlegung des Abstraktionsniveaus}
Beispiele:

\properparagraph{1) Aufzug}
%TODO Abb

\properparagraph{2) Tank}
(\textit{BB. AEH 1-3})

Abstraktion bedingt durch die AT-Aufgabe und/oder die Randbedingungen (z.B. Aktorik/Sensorik).

Also: Modellierung kombiniert Aufgaben(Ziele) und Prozesswissen(Physik, Randbedingungen)

\subsubsection{Wissensmodellierung und Verarbeitung}
(\textit{BB. AEH 1-4})

\subsection{Begriffsbestimmungen}
Erforderlich zur Modellformalisierung

zentrale Begriffe:
\begin{itemize}
	\item System 
	\item Zustand
	\item Ereignis 
	\smash{\raisebox{.5\dimexpr2\baselineskip+4\itemsep+2\parskip}{$\left.\rule{0pt}{.5\dimexpr3\baselineskip+3\itemsep+3\parskip}\right\}\text{Dynamisches System}$}}
\end{itemize}

(\textit{BB. AEH 1-5})

\properparagraph{Beispiel Tank (Dynamisches System)}

%TODO Abb.

\underline{Ereignisse}
\begin{itemize}
	\item \textbf{autonom}: (Größenänderung aufgrund interner Zustandsübergänge)
	\item \textbf{gesteuert}: (Größenänderung aufgrund externer Stellsignale)
\end{itemize}

\properparagraph{Beispiel Zulaufventil}

%TODO Abb.

\underline{In dieser Vorlesung}

Zwei besondere Ausprägungen dynamischer Systeme betrachtet:
\begin{enumerate}
	\item Ereignisdiskrete Systeme
	\smash{\raisebox{-.5\dimexpr1\baselineskip+4\itemsep+2\parskip}{$\left.\rule{0pt}{.5\dimexpr2\baselineskip+3\itemsep+3\parskip}\right\}\text{Dynamisches System}$}} 
	\item Hybride Systeme 
\end{enumerate}

\underline{Dynamisches System}

\begin{figure}[H]
	\centering
	\resizebox{!}{!}{\includegraphics[width=\textwidth]{img/2019_04_29_Abb5.tikz}}
\end{figure}

\properparagraph{Inhalte der Vorlesung}

\begin{itemize}
	\item Ereignisdiskrete Systeme: 
	\begin{itemize}
		\item Modelltypen und Prozessmodellierung
		\item Analyse und Simulation
		\item Spezifikation und Synthese von Steuerungen
	\end{itemize}
	\item Hybride Systeme
	\begin{itemize}
		\item Modellierung
		\item Simulation, Analyse, Steuerung
	\end{itemize} 
\end{itemize}


\subsection{Beispiel: gesteuerter Chargenprozess}
Prozessgeführte Ablaufsteuerung (Struktur siehe \textit{BB. AEH 1-6})

Prinzip: Einfüllen - Heizen - Entleeren einer Produktmenge(=Charge) [Füllhöhe $h$, Temperatur $\vartheta$] mittels der Stellgrößen [Ein- und Auslassventile $v_1$, $v_2$, Heizung $H$] durch eine diskrete Steuerung.




\end{document}

