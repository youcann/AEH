\marginpar{VL am 08.07.19}
\underline{Wiederholung}

\underline{$\wbar{x}(0) \neq \wbar{v}$}
\begin{equation}
	\forall k \ge l : \wbar{A}^{k+\gamma} = \lambda^\gamma \otimes \wbar{A}^k \Rightarrow \wbar{x}(k+\gamma) = \lambda^\gamma \otimes \wbar{x}(k)
\end{equation}

\underline{Sonderfall $\gamma=1$}

\begin{equation}
	\forall k \ge l : \wbar{A}^{k+1} = \lambda \otimes \wbar{A}^k
\end{equation}

\begin{align}
	(\ref{eq:mpa2}) \Rightarrow \wbar{x}(k+1) &= \wbar{A}^{k+1} \otimes \wbar{x}(0) \\
	&= \wbar{A}^k \otimes k \lambda \otimes \wbar{x}(0)
	&= \lambda \otimes \wbar{x}(k)
\end{align}
(also: $x_i(k+1) = \lambda + x_i(k)$)

außerdem: wegen $\wbar{x}(k+1) = \wbar{A} \otimes \wbar{x}(k)$ nach (\ref{eq:mpa1}):
\begin{equation}
	\wbar{A} \otimes \wbar{x}(k) = \lambda \otimes \wbar{x}(k)
\end{equation}
d.h. $\wbar{x}(k)$ : Eigenvektor $\wbar{v}$ von $\wbar{A}$

\begin{equation}
	\Rightarrow \wbar{x}(k+1) = \lambda \otimes \wbar{v}
\end{equation}

\subsubsection{Beispiel}
Beispiel zur Analyse des Zeitverhaltens \beiblatt{4-15, 4-16}



\section{Spezifikation und Entwurf von ereignisdiskreten Steuerungen}
Steuerungsentwurf $\hat{=}$ spezielle Automatisierungsaufgabe \beiblatt{5-1}

\underline{Ausgangspunkt}
\begin{itemize}
	\item DES-Modell des ungesteuerten Prozess ($\hat{=}$ Strecke)
	\item Spezifikation des Steuerungsziels (gesteuertes Prozessverhalten)
\end{itemize}

\beiblatt{5-2} (Batch-Prozess aus 1.3)

\underline{ProblemXXXX beim ereignisdiskreten Steuerungsentwurf}
\begin{itemize}
	\item Entwurfsmethodik
	\item Repräsentation (Modellierung)
	\item Realisierung (Implementierung)
	\item Verifikation
\end{itemize}

\underline{Ereignisdiskrete Steuerung}

Definition und Struktur \beiblatt{5-3}.

\subsection{Klassifikation von Steuerungszielen und Steuerungen}
\begin{itemize}
	\item Steuerungsziele
	\item Steuerungshierarchie
	\smash{\raisebox{.5\dimexpr1\baselineskip+2\itemsep+1\parskip}{$\left.\rule{0pt}{.5\dimexpr2\baselineskip+2\itemsep+2\parskip}\right\}\text{\beiblatt{5-4}}$}}
\end{itemize}

\subsubsection{Verrieglungssteuerungen}\label{subsec:verr}
häufig sicherheitsrelevante Prozessteile betroffen $\Rightarrow$ prozess-nahe Steuerung

\properparagraph{Beispiel: Motorsicherheitssteuerung}

%TODO Abb

\underline{Kennzeichen:}

Steuergrößen ergeben sich aus einer Verknüpfung der Eingangsgrößen (Verknüpfungssteuerung), die entweder 
\begin{enumerate}[label=(\alph*)]
	\item rein statisch (Schaltnetz) -oder-
	\item durch integrierte Speicher und Zeitdefinitionen dynamisch (Schaltwerke)
\end{enumerate}
ist.

\subsubsection{Ablaufsteuerungen}
Nächsthöhere Steuerung auf \textit{Maschinenebene}.

\underline{Kennzeichen:}

Steuergrößen ergeben sich aus dem schrittweisen Ablauf (sequentielle Steuerung) in Abhängigkeit von Übergangsbedingungen, die sich entweder
\begin{enumerate}[label=(\alph*)]
	\item zeitgeführt (Bsp. Ampel) -oder- \label{enum:ablaufA}
	\item prozessgeführt (Bsp. Eisenbahnschranke) -oder- \label{enum:ablaufB}
	\item als Kombination von \ref{enum:ablaufA} und \ref{enum:ablaufB} (Bsp. Batchprozess aus 1.3)
\end{enumerate}
durch Auswertung der Messgrößen ergeben.

\subsubsection{Koordinationssteuerungen}
Steuerungen auf \textit{Betriebsebene}.

\underline{Kennzeichen:}

Steuergrößen ergeben sich aus übergeordneten Vorgaben zur Optimierung des Prozessablaufs (z.B. Energie-, Zeit-, Geld-Optimierung) und dienen zur Koordination der Teilprozesse (Einzel-/Gruppensteuerung)

\properparagraph{Beispiel}
Stückzahlmaximierung in der Fertigung

\subsection{Steuerungsspezifikation}
Definition und Arten \beiblatt{5-5}

\subsubsection{Programmiersprachen nach IEC 1131-3}
Beispiel: Motorsicherheitssteuerung nach \ref{subsec:verr} \beiblatt{5-6, 5-7}

\subsubsection{Petri-Netze}
Im Prinzip äquivalent zu SFC
\begin{itemize}
	\item[+] Vorteil: Analysemöglichkeiten (vgl. Kapitel 4)
	\item[-] Nachteil: Im industriellen Umfeld bislang nur vereinzelt eingesetzt 
\end{itemize} 

Für Steuerung: Ein-/Ausgabekonzept erforderlich

$\Rightarrow$ (steuerungstechnisch) \underline{Interpretiertes Petri-Netz (IPN)}

%TODO Abb.








