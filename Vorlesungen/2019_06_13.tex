\marginpar{VL am 13.06.19}
	\begin{figure}[H]
		\centering
		\includegraphics{img/2019_06_13/abb1.tikz}
	\end{figure}
	
\properparagraph{Reversibilität:}
Def.: \beiblatt{4-5 unten} 

\underline{also}: je zwei Markierungen der Erreichbarkeitsmenge durch Schlatungsgrenzen ineinander überführbar.

Bsp.: s. $E_N$ in Bild oben

Petri-Netz nicht reversibel, da z.B. $M_1 = (1,1,0), M_2 = (0,2,0) \in R_N(M_0)$

$M_2 \in R_N(M_1)$, aber $M_1\notin R_N(M_2)$ 

\subsubsection{Lebendigkeit und Beschränktheit}
\properparagraph{Lebendigkeit}
Def.: \beiblatt{4-6}

Auswirkungen toter Transitionen auf die Netzdynamik: \underline{Verklemmungen} \beiblatt{4-6}

\textbf{Bsp.:} s. $E_N$ in Bild oben

Markierung $M_2$: keine Transition aktiviert $\rightarrow$ tote Markierung $\rightarrow$ totale Verklemmung möglich

\properparagraph{Beschränktheit}
Def.: \beiblatt{4-6}

\textbf{Bsp.:} s. $E_N$ in Bild oben

alle Stellen 2-beschränkt bei $M_0$ $\rightarrow$ $N$ 2-beschränkt bei $M_0$ $\rightarrow$ $N$ nicht sicher. 

\subsection{Analyse von Petri-Netzen}
\underline{bisher}: Definition von PN-Eigenschaften

\underline{jetzt}: Methoden zur formalen Analyse von PNen

\textbf{Zwei Analysearten}:

\begin{enumerate}
	\item{\textbf{graphentheoretische Analyse:}} erfolgt auf Basis des Erreichbarkeitsgraphen $E_N$ bzw. auf dessen Kondensation (s. später) 
	
	$\rightarrow$ notwendige und hinreichende Kriterien
	
	\item{\textbf{algebraische Analyse:}} erfolgt auf Basis der algebraischen Netzbeschreibung mittels sogenannter Netzunvarianten (s. später, im Prinzip Rückführung auf das Erreichbarkeitsproblem)
	
	$\rightarrow$ nur notwendige Kriterien
\end{enumerate}

\subsubsection{Graphentheoretische Analyse mit dem Erreichbarkeitsgraphen}
\textbf{Anforderung:} Beschränktes PN

\properparagraph{a) Unmittelbare Betrachtung von $E_N$}
Analyse möglich auf 

\begin{itemize}[align=left]
	\item{\textbf{Erreichbarkeit:}} \beiblatt{4-7}\\
	anwendbare Schaltungsgrenzen $\equiv$ mögliche Pfade im $E_N$
	\item{\textbf{Lebendigkeit:}} 
	
	\begin{itemize}
		\item \textbf{tote Transitionen: }  \beiblatt{4-7}\newline
		
			\textbf{Beispiel:}
			 \begin{figure}[H]
				\centering
				\includegraphics{img/2019_06_13/abb2.tikz}
			\end{figure}
		
			\begin{itemize}
			\item[-]{\textbf{a)}} $M_0 = (0,1,0,0)$
			 
			\begin{figure}[H]
				\centering
				\includegraphics{img/2019_06_13/abb3.tikz}
			\end{figure}
			
			\textbf{Erreichbarkeitsgraph}
			
			\begin{figure}[H]
				\centering
				\includegraphics{img/2019_06_13/abb4.tikz}
			\end{figure}
			$\rightarrow t_1$ tote Transition
			
			\item[-]{\textbf{b)}} $M_0 = (1,0,0,0)$
			\begin{figure}[H]
				\centering
				\includegraphics{img/2019_06_13/abb5.tikz}
			\end{figure}
			
			\textbf{Erreichbarkeitsgraph}
			
			\begin{figure}[H]
				\centering
				\includegraphics{img/2019_06_13/abb6.tikz}
			\end{figure}
			$\rightarrow$ $E_N$ mit neuem Knoten $\rightarrow$ $t_1$ nicht tot, aber $t_1$ auch nicht lebendig, da.....Rest wurde weggewischt (wahrscheinlich, weil man nicht mehr zum Zustand $(1,0,0,0)$ zurückkehren kann) %TODO drum kümmern, dass neue Zeile nicht so weit vorne anfängt
		\end{itemize}
		\item \textbf{totale Verklemmung:} \beiblatt{4-7}
	\end{itemize}	

	
	
	\item{\textbf{Konflikt:}} 
	\begin{itemize}
		\item möglich an Knoten (Markierungen) mit mehr als einer auslaufenden Kante
		\item tritt auf, wenn die auslaufenden Kanten der Folgeknoten nicht mehr mit den alternativen Transitionen beschriftet sind  %TODO drum kümmern, dass neue Zeile nicht so weit vorne anfängt
		Bsp.: s. Bild 2, Markierung bei $S_1$
		$\underline{m}_0^{T} = [1\, 0\, 0\,  1] \, , \, k(S_1) = 1$
		
		$\underline{m}_1^{T} = [0\, 1\, 0\,  1] \, , \, \underline{m}_2^{T} = [1\, 1\, 0\,  0] \, , \,\underline{m}_3^{T} = [0\, 0\, 1\,  1] \, , \, \underline{m}_4^{T} = [1\, 0\, 1\,  0] \, , \, \underline{m}_5^{T} = [0\, 1\, 1\,  0] $
		\begin{figure}[H]
			\centering
			\includegraphics{img/2019_06_13/abb7.tikz}
		\end{figure}
		
		potentielle Konflikte: bei $\underline{m}_0$ (vorhanden) und $\underline{m}_4$ (nicht vorhanden)
	\end{itemize}

\end{itemize}


\properparagraph{b) Genauere Analyse des $E_N$}
Analyse des PNs möglich auf
\begin{itemize}
	\item{\textbf{Reversibilität}} \beiblatt{4-7} 
	
	\textbf{hierzu erforderlich}: Überprüfung des \underline{starken Zusammenhangs} bzw. der \underline{Kondensation} des $E_N$
	
	\textbf{graphentheoretische Grundlagen} (s. BB AEH 4-1 bis 4-4)
	
	\textbf{Beispiel zur algebraischen Bestimmung einer Graphenkondensation}
	\begin{figure}[H]
		\centering
		\includegraphics{img/2019_06_13/abb8.tikz}
	\end{figure}
	
	Adjazenzmatrix $A = \left( \begin{array}{rrrr}
	0 & 0 & 0 & 0 \\
	1 & 0 & 1 & 0 \\
	1 & 0 & 0 & 1 \\
	0 & 0 & 1 & 0 \\
	\end{array}\right)$
	
	$
	A^{2} = \left( \begin{array}{rrrr}
	0 & 0 & 0 & 0 \\
	1 & 0 & 0 & 1 \\
	0 & 0 & 1 & 0 \\
	1 & 0 & 0 & 1 \\
	\end{array}\right)
	A^{3} = \left( \begin{array}{rrrr}
	0 & 0 & 0 & 0 \\
	0 & 0 & 1 & 0 \\
	1 & 0 & 0 & 0 \\
	0 & 0 & 1 & 0 \\
	\end{array}\right)
	A^{4} = \left( \begin{array}{rrrr}
	0 & 0 & 0 & 0 \\
	1 & 0 & 0 & 1 \\
	0 & 0 & 1 & 0 \\
	1 & 0 & 0 & 1 \\
	\end{array}\right)	
	 $
	 %TODO format

Mehr als $A^{4}$ macht keinen Sinn, weil bei vier Knoten die maximale Pfadlänge 4 beträgt.

\begin{equation}
	A^{T} = A \cup A^{2} \cup A^{3} \cup A^{4} = 
	\left( \begin{array}{rrrr}
	0 & 0 & 0 & 0 \\
	1 & 0 & 1 & 1 \\
	1 & 0 & 1 & 1 \\
	1 & 0 & 1 & 1 \\
	\end{array}\right)
\end{equation}

Reflexiv transitive Hülle:

\begin{equation}
	A^{\ast} =
	\left( \begin{array}{rrrr}
	1 & 0 & 0 & 0 \\
	1 & 1 & 1 & 1 \\
	1 & 0 & 1 & 1 \\
	1 & 0 & 1 & 1 \\
	\end{array}\right)
\end{equation}
\end{itemize}