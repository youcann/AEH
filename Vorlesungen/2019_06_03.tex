\marginpar{VL am 03.06.19}
\begin{enumerate}
	\setcounter{enumi}{1}
	\item \underline{Verwenden von Test-/Inhibitorkanten}
	\beiblatt{3-3 oben}
	%TODO beiblatt Abb
	
	\properparagraph{Kennzeichen:}
	\begin{itemize}
		\item Verknüpfungen übersichtlicher
		\item keine Schleifen
		\item Ursache/Folge-Kopplung
		\item Spezialkanten vorhanden
	\end{itemize}
	
	\item \underline{Modellierung als NCE-System}
	\beiblatt{3-3 unten}
	%TODO beiblatt Abb.
	
	\properparagraph{Kennzeichen:}
	\begin{itemize}
		\item transparent und übersichtlich
		\item echte Ereignisparalellität
		\item Spezialkanten vorhanden
	\end{itemize}
\end{enumerate}

\subsubsection{Fördereinrichtung: Modellierung als NCE-System}
Hier: Modellierung eines Hubtisches: \beiblatt{3-4}

\begin{enumerate}
	\item Modularisierung
	\begin{enumerate}
		\item \underline{Mechanik (Tischposition)}\\
		Zustandsgrößen:
		\begin{itemize}
			\item Position (oben/unten/Mitte)
			\item Positionsänderung (auf/ab)
		\end{itemize}
		\item \underline{Motor (Aktorik)}\\
		Zustandsgrößen:
		\begin{itemize}
			\item Antriebsrichtung (hoch/runter)
			\item Motorzustand (aus/Überlast)
		\end{itemize}
		\item \underline{Endschalter}\\
		Zustandsgrößen:
		\begin{itemize}
			\item Betätigungen (0/1)
		\end{itemize}
	\end{enumerate}
	\item Verkopplung der Module
	\beiblatt{3-5}
\end{enumerate}

\subsection{Ressourcenorientierte Modellierung}
häufig angewandt bei wesensmäßig sequentiellen/Objektbezogenen Prozessvorgängen (z.B. Fertigungstechnik, Fördertechnik, Verkehrstechnik).

Mit \textit{Ressourcen} (z.B. Bearbeitungsmaschinen, Kräne, Gleisabschnitte), für die ein diskreter Belegungsablauf durch Nutzer (Teile, Lasten, Züge) existiert.

\properparagraph{Vorgehensweise:}
\begin{itemize}
	\item Bestimmung der relevanten Ressourcen im Prozessgeschehen (PN: Stellen) und deren mögliche Nutzer (PN: Marken)
	\item Festlegung der Ressourcen-verkopplung und -abfolge (PN: Kanten) mit entsprechenden Nutzerwechseln (PN: Transitionen)
	\item ggf. Ressourcen zusammenfassen (PN: Teilnetze) 
\end{itemize}

\properparagraph{Ergebnis:}
Bei PNen häufig S/T-Netze, da z.B. zur Abbildung mehrerer Nutzer (mehrere Marken) $K(s_i)>1$

\subsubsection{Flugplatzgeschehen: Modellierung als PN}
Hier: Ausschnitt einer PN-Modellierung der Flugzeugbewegungen auf einem Flugplatz:

\begin{itemize}
	\item Luftraum:\\
	\begin{itemize}
		\item An- und Abflug
		\item Warteschleife
	\end{itemize}
	
	\item Boden:\\
	\begin{itemize}
		\item Start- und Landebahnen
		\item Rollbahn
	\end{itemize}
\end{itemize}

\beiblatt{3-6}

Typische Steuerungsaufgaben:
\begin{itemize}
	\item Ablaufkoordination
	\item Konfliktlösung
	\item Verhinderung verbotener Zustände
\end{itemize}

\section{Analyse ereignisdiskreter Systeme}
Hier: als DES-Modelle speziell PN betrachtet.

Analyse für unterschiedliche Modelltypen:
\begin{itemize}
	\item Streckenmodell (ungesteuerter Prozess)
	\item Modell der Steuerung
	\item Modell des gesteuerten Prozess
\end{itemize}

Erforderliche Grundlagen: Elemente der Graphentheorie \beiblatt{4-1 \dots 4-4}

\subsection{Eigenschaften von PNen}
Hier: konkrete Eigenschaften von PNen, die zur Analyse des Modellverhaltens dienen. Also z.B. Auffinden von Modellzuständen, die

\begin{itemize}
	\item unmöglich sind.
	\item verboten sind.
	\item nicht mehr verlassen werden können.
\end{itemize}

\subsubsection{Erreichbarkeit und Reversibilität}
Erreichbarkeit: \beiblatt{4-5}

also: Erreichbarkeitsgraph $E_N$: grafische Übersicht über alle von $M_0$ (über anwendbare Schaltsequenzen) erreichbare Markierungen.

\properparagraph{Beispiel:}

\begin{equation}
	\wbar{K} = \m{2\\2\\2}, \qquad M_0 = (2,\,0,\,0)
\end{equation}

\begin{figure}[H]
	\centering
	\subfloat[PN]{\includegraphics[width=0.3\textwidth]{img/2019_06_03/abb1.tikz}} 
	\qquad 
	\subfloat[$E_N$]{\includegraphics[width=0.6\textwidth]{img/2019_06_03/abb2.tikz}}
\end{figure}

ablesbar: XXX Pfade im $E_N$: mögliche Konflikte zwischen Transitionen möglich 

im Beispiel:
\begin{itemize}
	\item Bei $M_3$: Konflikt zwischen $t_1$ und $t_2$
	\item Bei $M_3$: Kein Konflikt zwischen $t_1$ und $t_3$
\end{itemize}










