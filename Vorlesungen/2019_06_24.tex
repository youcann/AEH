\marginpar{VL am 24.06.19}
\properparagraph{$E_G$:}

\begin{figure}[H]
	\centering
	\includegraphics{img/2019_06_24/abb1.tikz}
\end{figure}

\begin{equation}
	A^\ast=
	\begin{bmatrix}
	\textcolor{red}{1} & 0 & 0 & 0 \\
	1 & \textcolor{blue}{1} & 1 & 1 \\
	1 & 0 & \textcolor{green}{1} & 1 \\
	1 & 0 & 1 & 1 \\
	\end{bmatrix}
\end{equation}

\properparagraph{Auswertung:}
%TODO k=K, was ist was?

\begin{enumerate}[label=(\alph*)]
	\item \textcolor{red}{$a_{11}^\ast =1$}, $a_{1i}^\ast = 0 \neq 1 = a_{i1}^\ast: \circled{1}=K_1^k$
	\item \textcolor{blue}{$a_{22}^\ast = 1$}, $a_{2i}^\ast = 1 \neq 0 = a_{i2}^\ast: \circled{2}=K_2^k$
	\item \textcolor{green}{$a_{33}^\ast = 1$}, $a_{3i}^\ast = a_{i3}^\ast = 1: \circled{3}+\circled{4} = K_3^k$
	\item $a_{21}^\ast=1$, $a_{12}^\ast=0: \circled{2} \rightarrow \circled{1}$
	\item $a_{31}^\ast=1$, $a_{13}^\ast=0$, $a_{41}^\ast=1$, $a_{14}^\ast=0: \circled{3} \rightarrow \circled{1} \leftarrow \circled{4}$
	\item $a_{23}^\ast=1$, $a_{32}^\ast=0$, $a_{24}^\ast=1$, $a_{42}^\ast=0: \circled{3} \leftarrow \circled{2} \rightarrow \circled{4}$
\end{enumerate}

\properparagraph{$E_G^k:$}
\begin{figure}[H]
	\centering
	\includegraphics{img/2019_06_24/abb2.tikz}
\end{figure}

\properparagraph{Beispiel zur grafischen KondensXXXermittlung}
\beiblatt{4-4}

$E_N$ im Beispiel-Netzwerk

\underline{PN:}

\begin{figure}[H]
	\centering
	\includegraphics[width=0.6\textwidth]{img/2019_06_24/abb3.tikz}
\end{figure}

\underline{$E_N$:}

\begin{figure}[H]
	\centering
	\includegraphics[]{img/2019_06_24/abb4.tikz}
\end{figure}


\underline{$E_N^k$:}

\begin{figure}[H]
	\centering
	\includegraphics[]{img/2019_06_24/abb5.tikz}
\end{figure}

also: PN so nicht reversibel, da $E_N$ nicht stark zusammenhängend.

\underline{Lebendigkeit}
\beiblatt{4-7}

In Senke $K_2^k$ Transition $t_1$ nicht enthalten $\Rightarrow$ PN nicht lebendig.

\subsubsection{Algebraische Analyse mit S- und T-Invarianten}
Ausgangspunkt: Algebraische PN-Beschreibung \beiblatt{2-12}

\begin{equation}\label{eq:mkplus1}
	\wbar{m}(k+1) = \wbar{m}(k) + \wbar{N} v(k+1)
\end{equation}

\underline{Invarianten:} Kennzeichnen Mengen von PN-Elementen, die unabhängig (invariant) von der Anfangsmarkierung aufgrund der PN-Topologie bestimmte Eigenschaften besitzen.

\properparagraph{1) S-Invarianten}
Definition \beiblatt{4-8}

Herleitung: aus (\ref{eq:mkplus1})

Annahme: $\wbar{i}_s$ sei S-Invariante

\begin{equation}
	(\wbar{m}(k+1)-\wbar{m}(k))^T \wbar{i}_s= \underbrace{(\wbar{N} \cdot \wbar{v}(k+1))^T}_{\wbar{v}^T(k+1) \underbrace{\wbar{N}^T \wbar{i}_s}_{=0 \text{ (Def.)}}}
\end{equation}

Für jede S-Invariante $i_S$ gilt also:
\begin{equation}
	\wbar{m}(k+1)^T \wbar{i}_s = \wbar{m}(k)^T \wbar{i}_s, \qquad \forall k=0,\,1,\,2,\,\ldots
\end{equation}

\begin{equation}
\Rightarrow \wbar{m}(k)^T \wbar{i}_s = \wbar{m}(0)^T \wbar{i}_s = \text{const.}
\end{equation}

\underline{Beispiel:}

\begin{figure}[H]
	\centering
	\includegraphics[width=0.6\textwidth]{img/2019_06_24/abb6.tikz}
\end{figure}

\begin{equation}
	\wbar{N} = 
	\m{
	-1 & 1 \\
	-1 & 1 \\
	1 & -1 	
	}
\end{equation}

S-Invarianten:
\begin{equation}
	\wbar{N}^T \wbar{i}_s = 0 =
	\m{
	-1 & -1 & 1\\
	1  &  1 & -1	
	}
	\m{
	i_{s1}\\
	i_{s2}\\
	i_{s3}	
	}
\end{equation}

\begin{align}
	-i_{s1} - i_{s2} + i_{s3} &= 0\\
	i_{s1} + i_{s2} - i_{s3}  &= 0
\end{align}
Linear abhängiges Gleichungssystem unterbestimmt (unendlich viele Lösungen)

Ganzzahlige Lösungen gesucht, zum Beispiel:
\begin{itemize}
	\item $i_{s1}=0 \Rightarrow i_{s3}=i_{s2} \text{(z.B. zu 1 gewählt)}$
	\begin{equation}
		\wbar{i}_s^I = \m{0\\1\\1}
	\end{equation}
	\item $i_{s1}=1 \Rightarrow i_{s3}=i_{s2}+1 \text{(z.B. $i_{s2}=0$ gewählt)}$
	\begin{equation}
	\wbar{i}_s^{II} = \m{1\\0\\1}
	\end{equation}
\end{itemize}

Bedeutung der S-Invarianten:

Zum Beispiel $\wbar{i}_{s}^I$:
\begin{equation}
	\wbar{m}^T \cdot \wbar{i}_s^I = M(s_2) + M(s_3) = M_0(s2) + M_0(s_1) = 1 =\text{const} \qquad \forall k
\end{equation}

Analog $\wbar{i}_{s}^{II}$:
\begin{equation}
	M(s_1) + M(s_3) = 1 = \text{const} \qquad \forall k
\end{equation}

Anmerkung: Auch jede Linearkombination linear unabhängiger S-Invarianten ist S-Invariante:
\begin{equation}
	\wbar{i}_s = c_1 \, \wbar{i}_{s}^{I} + c_2 \, \wbar{i}_{s}^{I}
\end{equation}

Zum Beispiel ($c_1=1,\,c_2=-1$):
\begin{equation}
	\wbar{i}_{s}^{III} = \m{0\\1\\1} - \m{1\\0\\1} = \m{-1\\1\\0}
\end{equation}

\begin{equation}
	\wbar{m}^T \wbar{i}_{s}^{III} = -M(s_1) + M(s_2) = 0
\end{equation}
d.h.
\begin{equation}
	M(s_1) = M(s_2), \qquad \forall k
\end{equation}

\underline{PN-Analyse mit S-Invarianten: Beschränktheit}
\beiblatt{4-8}

Beweis: Annahme $\wbar{i}_s > \wbar{0}$ existiert
\begin{equation}
	\Rightarrow \wbar{m}^T(k) \, \wbar{i}_s = \wbar{m}^T(0) \, \wbar{i}_s
\end{equation}
\begin{equation}
	m_i(k) \, i_{s,i} \le \wbar{m}^T(0) \, \wbar{i}_s
\end{equation}
\begin{equation}
	\Rightarrow m_i(k) \le \underbrace{\frac{\wbar{m}^T(0) \, \wbar{i}_s}{i_{s,i}}}_{K}
\end{equation}

Da für alle Stellen $s_i$ gültig $\Rightarrow$ Netz beschränkt!

Zum Beispiel:
\begin{equation}
	\wbar{i}_s^{III} = \m{1\\1\\2} \ge 0 \qquad (c_1=c_2=1), \; (\wbar{m}_0^T=\m{0 &0 &1})
\end{equation}

\begin{align}
M(s_1) & \le \frac{2}{1} = 2 \\[1em]
M(s_2) & \le \frac{2}{1} = 2 \\[1em]
M(s_3) & \le \frac{2}{2} = 1 
\end{align}

\properparagraph{2) T-Invarianten}
Definition \beiblatt{4-8}

Herleitung: aus (\ref{eq:mkplus1})

Annahme: $\wbar{i}_T$ sei T-Invariante

\begin{equation}
	\wbar{m}(k+1) = \wbar{m}(0) + \wbar{N} \underbrace{\left( \wbar{v}(1) + \wbar{v}(2) + \ldots + \wbar{v}(k+1) \right)}_{\wbar{i}_T}
\end{equation}

\begin{equation}
	\Rightarrow \wbar{m}(k+1) - \wbar{m}(0) = \underbrace{\wbar{N} \, \wbar{i}_T}_{=0}
\end{equation}

\begin{equation}
	\Rightarrow \text{T-Invariante } \wbar{i}_T:\; \wbar{m}(k+1) = \wbar{m}(0) \qquad \forall k
\end{equation}
(nach gewisser Schaltsequenz $\wbar{N} \, \wbar{i}_T$)

Beispiel \beiblatt{4-9, 4-10}

PN-Analyse mit T-Invarianten: Reversibilität, Lebendigkeit \beiblatt{4-9}




