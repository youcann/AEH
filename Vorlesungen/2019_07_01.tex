\marginpar{VL am 01.07.19}
\subsection{Analyse zeitbewerteter Synchronisationsgraphen mit der Max-Plus-Algebra}
\underline{Beispiel:} Batchprozess aus 1.3

\begin{figure}[H]
	\centering
	\includegraphics[width=0.6\textwidth]{img/2019_07_01/abb1.tikz}
\end{figure}

\underline{Merkmale}
\begin{itemize}
	\item Zyklischer Ablauf
	\item Synchronisationen
	\item Konflikte aufgelöst (keine Alternativen)
	\item Zeitbewertungen
\end{itemize}

$\Rightarrow$ Modellierung mit zeitbewerteten Synchronisationsgraphen möglich!

Schaltvorgang \beiblatt{4-11}

\subsubsection{Grundlagen der Max-Plus-Algebra (MPA)}
Max-Plus-Algebra: algebraische Struktur mit der Grundmenge $R_{max}=\mathbb{R} \cup \{\underbrace{-\infty}_{\epsilon}\}$

Verknüpfungen:
\begin{itemize}
	\item Addition $\oplus$
		$a \oplus b := \text{max}(a,b)$
	\item Multiplikation $\otimes$
		$a \otimes b := a + b$
\end{itemize}

Eigenschaften: \beiblatt{4-12}

\underline{Anmerkungen:}
\begin{itemize}
	\item Bei Matrizen häufig kurz $\ubar{A} \, \ubar{B}$ statt $\ubar{A} \otimes \ubar{B}$ geschrieben
	\item Gleichungssysteme mit der MPA nicht so einfach lösbar wie gewohnt, da inverses Element zu $\oplus$ (Maximumbildung) fehlt.
\end{itemize}

\subsubsection{Algebraische Beschreibung zeitbewerteter Synchronisationsgraphen}
Gleichung für Schaltzeitpunkte in der MPA: \beiblatt{4-13}

\underline{Beispiel}

\begin{figure}[H]
	\centering
	\includegraphics[width=0.6\textwidth]{img/2019_07_01/abb2.tikz}
\end{figure}

\begin{equation}
	\ubar{x}(k+1) = 
	\underbrace{
		\m{
		\epsilon & \epsilon & \epsilon & \epsilon \\
		\tau_{21} & \epsilon & \epsilon & \epsilon \\
		\tau_{31} & \epsilon & \epsilon & \epsilon \\
		\epsilon & \tau_{42} & \tau_{43} & \epsilon \\
		}
	}_{\wbar{A}_0}
	x(k+1) 
	\oplus
	\underbrace{
		\m{
			\epsilon & \epsilon & \epsilon & \tau_{14} \\
			\epsilon & \epsilon & \epsilon & \epsilon \\
			\epsilon & \epsilon & \epsilon & \epsilon \\
			\epsilon & \epsilon & \epsilon & \epsilon \\
		}
	}_{\wbar{A}_1}
\end{equation}

\properparagraph{Herleitung der expliziten Gleichung (rekursiv)}

\begin{equation}
	\underline{x}(k+1) = \wbar{A}_0 \wbar{x}(k+1) \oplus \wbar{A}_1 \wbar{x}(k)
\end{equation}

\begin{center}
	$\Downarrow$ sukzessive $(n-1)$-mal in sich selbst eingesetzt
\end{center}

\begin{align}
	\underline{x}(k+1) &= \wbar{A}_0^2 \wbar{x}(k+1) 
	\oplus \wbar{A}_0 \wbar{A}_1 \wbar{x}(k)
	\oplus \wbar{A}_1 \wbar{x}(k) \\
	&\vdots \nonumber \\
	&= \wbar{A}_0^n \wbar{x}(k+1) 
	\oplus \underbrace{\left( \wbar{A}_0^{n-1} \oplus \ldots \oplus \wbar{A}_0 \oplus \wbar{I} \right)}_{\wbar{A}_0^\ast} \wbar{A}_1 \wbar{x}(k)
\end{align}

Es gilt für $\wbar{A}_0^k$ (ohne Herleitung):

\begin{equation}
	\left( \wbar{A}_0^k \right)_{i,j} = 
	\left\{
	\begin{array}{cl}
		\text{Maximum aller Refgeguzahnf} & wann???\\
		\text{der Pfade mit k anfangs}\\
		\text{unmarkierten Stellen}\\
		\text{zwischen~} t_i \text{~und~} t_j \\
		\epsilon & \text{sonst}
	\end{array}
	\right.
\end{equation}

\underline{Beispiel}
\begin{equation}
	\wbar{A}_0^2 =
	\begin{bmatrix}
		\epsilon \\
		\epsilon & \epsilon \\
		\tau_{42} \otimes \tau_{21} \oplus \tau_{43} \otimes \tau_{21} & \epsilon & \epsilon & \epsilon\\
	\end{bmatrix}
\end{equation}

bei $n$ Transitionen muss $\wbar{A}_0^n = \wbar{N}$

\begin{equation}
	\Rightarrow \wbar{x}(k+1) = \underbrace{\wbar{A}_0^\ast \wbar{A}_1}_{=:\wbar{A}} \wbar{x}(k) = \wbar{A} \wbar{x}(k)
\end{equation}

\properparagraph{Beschreibung der Dynamik der Ereigniszeitpunkte}
($\wbar{x}(0)$ bekannt $\Rightarrow$ $\wbar{x}(k)$ bekannt $\forall k$)

\subsubsection{Petri-Netz-Analyse mit der MPA}
MPA: 
\begin{equation}
\label{eq:mpa1}
	\wbar{x}(k+1) = \wbar{A} \otimes \wbar{x}(k)
\end{equation}

Analyse auf Basis der Eigenwerte und Eigenvektoren von $\wbar{A}$ \beiblatt{4-14}.

aus \ref{eq:mpa1}:
\begin{align}
	\wbar{x}(1) &= \wbar{A} \otimes \wbar{x}(0) \\
	\wbar{x}(2) &= \wbar{A} \otimes \wbar{x}(1) = \wbar{A}^2 \otimes \wbar{x}(0) \\
	            &\vdots \nonumber \\
	\wbar{x}(k) &= \wbar{A}^k \otimes \wbar{x}(0) \label{eq:mpa2}
\end{align}

also: $\wbar{x}(k)$ abhängig von $\wbar{x}(0)$ und $\wbar{A}$ bzw. $\wbar{A}^k$.

\underline{Einfluss von $\wbar{x}(0)$}
\begin{enumerate}[label=(\alph*)]
	\item $\wbar{x}(0)=\wbar{v}$:\\
		\begin{align}
			(\ref{eq:mpa2}) \Rightarrow \wbar{x}(1) &= \wbar{A} \otimes \wbar{v} = \lambda \otimes \wbar{v} \\
			\wbar{x}(2) &= \wbar{A} \otimes \wbar{x}(1) = \wbar{A} \otimes \lambda \otimes \wbar{v} = \lambda^2 \otimes \wbar{v} \\
			&\vdots \nonumber \\
			\wbar{x}(k) &= \lambda^k \otimes \wbar{v} = \lambda^k \otimes \wbar{x}(0) \qquad (\forall k \ge 0)
		\end{align}
		(also: $x_i(k)=x_i(0)+\lambda k$)
	
	\item $\wbar{\wbar{x}}(0)\neq\wbar{v}$:\\
		Abhängig von $\wbar{A}$ ergibt sich ein Einschwingvorgang (Dauer $L$), mit dem sich ein $\lambda$ und ein $\gamma$ bestimmen lassen, so dass gilt:
		\begin{equation}
			\forall k \ge L : \wbar{A}^{k + \gamma} = \lambda^\gamma \otimes \wbar{A}^k
		\end{equation}
		
		\begin{align}
			\wbar{x}(k+\gamma) &= \wbar{A}^{k+\gamma} \otimes \wbar{x}(0)\\
			            &= \lambda^\gamma \otimes \wbar{A}^k \otimes \wbar{x}(0) \\
			            &= \lambda^\gamma \otimes \wbar{x}(k)
		\end{align}
		(also: $x_i(k+\gamma) = \gamma \lambda + x_i(k)$)
		($\gamma$: Zyklizität)
\end{enumerate}















