\subsection{Analyse zeitbewerteter Synchronisationsgraphen mit der Max-Plus-Algebra}
\underline{Beispiel:} Batchprozess aus 1.3

%TODO Abb

\underline{Merkmale}
\begin{itemize}
	\item Zyklischer Ablauf
	\item Synchronisationen
	\item Konflikte aufgelöst (keine Alternativen)
	\item Zeitbewertungen
\end{itemize}

$\Rightarrow$ Modellierung mit zeitbewerteten Synchronisationsgraphen möglich!

Schaltvorgang \beiblatt{4-11}

\subsubsection{Grundlagen der Max-Plus-Algebra (MPA)}
Max-Plus-Algebra: algebraische Struktur mit der Grundmenge $R_{max}=\mathbb{R} \cup \{\underbrace{-\infty}_{\epsilon}\}$

Verknüpfungen:
\begin{itemize}
	\item Addition $\oplus$
		$a \oplus b := \text{max}(a,b)$
	\item Multiplikation $\otimes$
		$a \otimes b := a + b$
\end{itemize}

Eigenschaften: \beiblatt{4-12}

\underline{Anmerkungen:}
\begin{itemize}
	\item Bei Matrizen häufig kurz $\ubar{A} \, \ubar{B}$ statt $\ubar{A} \otimes \ubar{B}$ geschrieben
	\item Gleichungssysteme mit der MPA nicht so einfach lösbar wie gewohnt, da inverses Element zu $\oplus$ (Maximumbildung) fehlt.
\end{itemize}

\subsubsection{Algebraische Beschreibung zeitbewerteter Synchronisationsgraphen}
Gleichung für Schaltzeitpunkte in der MPA: \beiblatt{4-13}

\underline{Beispiel}

%TODO Abb.

\begin{equation}
	\ubar{x}(k+1) = 
	\underbrace{
		\m{
		\epsilon & \epsilon & \epsilon & \epsilon \\
		\tau_{21} & \epsilon & \epsilon & \epsilon \\
		\tau_{31} & \epsilon & \epsilon & \epsilon \\
		\epsilon & \tau_{42} & \tau_{43} & \epsilon \\
		}
	}_{\underline{A_0}}
	x(k+1) 
	\oplus
	\underbrace{
		\m{
			\epsilon & \epsilon & \epsilon & \tau_{14} \\
			\epsilon & \epsilon & \epsilon & \epsilon \\
			\epsilon & \epsilon & \epsilon & \epsilon \\
			\epsilon & \epsilon & \epsilon & \epsilon \\
		}
	}_{\underline{A_1}}
\end{equation}

\properparagraph{Herleitung der expliziten Gleichung (rekursiv)}

\begin{equation}
	\underline{x}(k+1) = \underline{A_0} \underline{x}(k+1) \oplus \underline{A_1} \underline{x}(k)
\end{equation}

\begin{center}
	$\Downarrow$ sukzessive $(n-1)$-mal in sich selbst eingesetzt
\end{center}

\begin{align}
	\underline{x}(k+1) &= \underline{A_0^2} \underline{x}(k+1) 
	\oplus \underline{A_0} \underline{A_1} \underline{x}(k)
	\oplus \underline{A_1} \underline{x}(k) \\
	&\vdots\\
	&= \underline{A_0^n} \underline{x}(k+1) 
	\oplus \underbrace{\left( \underline{A_0^{n-1}} \oplus \ldots \oplus \underline{A_0} \oplus \underline{I} \right)}_{\underline{A_0^\ast}} \underline{A_1} \underline{x}(k)
\end{align}













