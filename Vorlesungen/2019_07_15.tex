\marginpar{VL am 15.07.19}
\subsection{Steuerungsentwurf}
\subsubsection{Entwurf von Ablaufsteuerungen}
Meist nur intuitives Vorgehen gemäß gewünschtem Prozessablauf \beiblatt{5-8}.

\properparagraph{Beispiel}
\begin{itemize}
	\item Batch-Prozess aus 1.3
	\item siehe Abschnitt 5.5.1
\end{itemize}

\subsubsection{Entwurf von Verriegelungssteuerungen}
Verriegelung: Verhindern verbotener Zustände \beiblatt{5-8 unten}

\properparagraph{PN:}
Verhinderung verbotener Markierungen

Beispiel: Flugplatz \beiblatt{3-6}
$\Rightarrow \text{mit: } m(k) \quad m_{11}+m_{5}+m_{6}+m_{7} \le 1$

Zwei Entwurfsverfahren:
\begin{enumerate}
	\item \underline{Entwurf mit Hilfe von S-Invarianten [Moody/AnXXX]}
		
		\properparagraph{Ausgangspunkt}
		\begin{itemize}
			\item PN-Modell des ungesteuerten Prozess: $\wbar{N} \wbar{m}(0) = \wbar{m}_0$
			\item $p$ Verbote für jeweils eine maximale Markenanzahl
			\begin{equation}
				\label{eq:pVerbote}
				\underbrace{\wbar{L} \wbar{m}(k)}_{
				\begin{subarray}{c}
					\underset{(p,n)}{\wbar{L}}  \text{Spezifikationen der~} \\ p \text{~Markierungen}
				\end{subarray}} \le \underbrace{\wbar{b}}_{
				\begin{subarray}{c}
					\text{maximal zulässigen} \\
					\text{Markenanzahlen in den} \\
					\text{spezifizierten Stellenmengen}
				\end{subarray}	
				} \quad( \forall k) 
			\end{equation}
			
			\properparagraph{Prinzip des Steuerungsentwurfs}
			Einfügen von $p$ zusätzlichen Stellen (Kontrollstellen) $s_{c1},\,\ldots,\,s_{sp}$ mit entsprechenden Kanten zur Erzielung von (\ref{eq:pVerbote}).
			
			\properparagraph{Vorgehen}
			\begin{enumerate}[label=(\roman*)]
				\item Umformung von (\ref{eq:pVerbote}) in eine Gleichung mit 
				\begin{equation}
				\wbar{m}_c(k) = \m{m_{c1}(k),\,\ldots,\,m_{cp}(k)}^T \qquad \text{($\hat{=}$ Schlupfvariablen)}
				\end{equation}
				
				\begin{equation}
					\label{eq:vorgehen1}
					\wbar{L} \wbar{m}(k) + \wbar{m}_c(k) = \wbar{b}
				\end{equation}
				
				\begin{equation}
					\label{eq:vorgehen2}
					\m{\wbar{L} & \wbar{I}} \m{\wbar{m}(k) \\ \wbar{m}_c(k)} = \wbar{b} 
					\; \text{bzw.} \;
					\m{\wbar{m}^T(k) & \wbar{m}_c^T(k)} \m{\wbar{L}^T \\ \wbar{I}} = \wbar{b}^T
				\end{equation}
				
				\item (\ref{eq:vorgehen2}) gültig $\forall k$:
				\begin{equation}
					\m{\wbar{m}^T(k) & \wbar{m}_c^T(k)} \m{\wbar{L}^T \\ \wbar{I}} 
					\stackrel{!}{=}
					\m{\wbar{m}^T(k+1) & \wbar{m}_c^T(k+1)} \m{\wbar{L}^T \\ \wbar{I}} 
				\end{equation}
				

				$\Rightarrow \m{\wbar{L}^T \\ \wbar{I}}$ : S-Invariante des erweiterten Petri-Netztes 
				
				mit $\wbar{N}_{erw} = \m{\wbar{N} \\ \wbar{N}_c}$, $\wbar{N}_c$: Kontrollstellen
				
				\item also gilt:
				\begin{equation}
					\wbar{N}_{erw}^T \cdot \m{\wbar{L}^T \\ \wbar{I}} \stackrel{!}{=} \wbar{0}
				\end{equation}
				
				\begin{equation}
					\m{N^T & N_c^T} \m{\wbar{L}^T \\ \wbar{I}} = \wbar{0}
					\; \text{bzw.} \;
					N_c^T = - N^T L^T
				\end{equation}
				
				\begin{equation}
					\wbar{N}_c = - \wbar{L} \, \wbar{N} \; \text{(Festlegung der zusätzlichen Kanten für die Steuerung)}
				\end{equation}
				
				\item außerdem: (\ref{eq:vorgehen1}) $\Rightarrow$
				\begin{equation}
					\wbar{m}_c(0) = \wbar{b} - \wbar{L} \, \wbar{m}(0) \; \text{(Anfangsmarkierung der Stellen)}
				\end{equation}
			\end{enumerate}
			
			\properparagraph{Anmerkung:}
			Im gesteuerten PN sind neue tote Markierungen möglich. Daher ist stets eine erneute Überprüfung des neuen Erreichbarkeitsgraphen erforderlich.
			
			
		\end{itemize}
	\item \underline{Entwurf mit Hilfe der Max-Plus-Algebra}
	
	\properparagraph{Ausgangspunkt}
	\begin{itemize}
		\item Zeitbewerteter Synchronisationsgraph der Strecke $\wbar{A}$
		\item Verbote gleichzeitiger Stellenbelegungen, z.B. Stellen $s_i$ und $s_j$.
	\end{itemize}
	
	\properparagraph{Prinzip des Steuerungsentwurf}
	Gezielte Beeinflussung der Transitionenschaltzeitpunkte in den Vorbereichen von $s_i$ und $s_j$ ($\hat{=}$ Veränderung der Matrix $\wbar{A}$) durch zusätzliche Stellen, z.B.
	\begin{equation}
		t_r = \bullet s_i \; \text{und} \; t_s = \bullet s_j
	\end{equation}
	
	Entwurfsforderung:
	\begin{equation}
		x_r(k) \neq x_s(k+l) \qquad (\forall k,l \ge 0)
	\end{equation}
\end{enumerate}

\subsubsection{Verifikation des gesteuerten Prozessverhaltens}
\beiblatt{5-9}

\subsubsection{Implementierung}
Zum Beispiel mit einer SPS \beiblatt{5-10}

\subsection{Beispiele}
\subsubsection{Steuerung eines Hubtisches}
Gegeben: NCE-Modell der Strecke \beiblatt{3-5}

Hier: Entwurf einer Ablaufsteuerung mit integrierter Verriegelungssteuerung (intuitiver Entwurf)

\properparagraph{Spezifikationen}
\begin{itemize}
	\item \textbf{Ablauf}
	\begin{figure}[H]
		\centering
		\includegraphics{img/2019_07_15/abb1.tikz}
	\end{figure}
	\item \textbf{Verriegelung}
	\begin{itemize}
		\item Hubtischbewegung ist verboten, falls ein Bediener innerhalb des Arbeitsbereichs ist
		\item Umsetzung der Spezifikation im Steuerungsmodul
		\item Verknüpfung mit der Strecke \beiblatt{5-11}
		\item Modell des gesteuerten Prozess \beiblatt{5-12}
	\end{itemize}
\end{itemize}



