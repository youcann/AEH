\marginpar{VL am 22.07.19}
\subsubsection{Steuerung einer Fertigungsanlage}

\properparagraph{PN-Modell einer Werkstückbearbeitung}
\beiblatt{5-11}

\begin{itemize}
	\item Erreichbarkeitsgraph (komprimiert) \beiblatt{5-14}
	\item Tote Markierung $\wbar{m}_{tot}$ vorhanden ($\hat{=}$ Maschine vollständig belegt und neuer Ladevorgang aktiviert)
\end{itemize}

\begin{enumerate}
	\item \underline{Entwurf mittels der S-Invarianten \beiblatt{5-15}}
	%TODO BB
	\item \underline{Entwurf mittels der Max-Plus-Algebra}
	Alternatives Prozessmodell (Synchronisationsgraph) und Darstellung der MPA: \beiblatt{5-16}
	
	Intuitiver Steuerungsentwurf: \beiblatt{5-17}
\end{enumerate}


\section{Hybride Systeme}
Hybride Systeme: \beiblatt{1-5}

\subsection{Hybride Phänomene}
\underline{allgemein:} diskrete Ereignisse im kontinuierlichen Systemverlauf

Übersicht: \beiblatt{6-1}

\properparagraph{Beispiele}
\underline{Umschalten}
\begin{itemize}
	\item selbstständig: an Systemgrenzen (=Triggermengen), z.B. Tanküberlauf
	\item gesteuert: z.B. Ventilöffnung
\end{itemize}

\underline{Sprung}
\begin{itemize}
	\item selbstständig: z.B. Hystereseeffekte
	\item gesteuert: z.B. el. Schalter
\end{itemize}

\subsection{Das Netz-Zustands-Modell (NZM)}
Beschreibungsform für hybride Systeme, die modularer zwei statisch verkoppelte Systemanteile enthält.

\begin{itemize}
	\item kontinuierlicher Systemanteil: Erweitertes Zustandsmodell (EZM)
	\begin{itemize}
		\item Zustands-DGL $\wbar{\dot{x}}_K(t) = f(\wbar{x}_K(t),\,\underbrace{\wbar{u}_K(t),\,\wbar{v}_K}_{\text{Eingänge}})$
		\item Impulsgleichung $x_K(t_K) = g(x_K(t_k^-),\,v_K(t_k),\,v_K(t_k^-))$
	\end{itemize}
	\item ereignisdiskreter Systemanteil: Interpretiertes PN (IPN)
	\begin{itemize}
		\item Netzzustand ($\hat{=}$ Markierung) $x_D(k) = \wbar{h}(x_D(k-1),\,b(k-1))$
		\item Transitionenschaltbedingung $b(k) = l(\underbrace{u_D(k),\,v_D}_\text{Eingänge})$
	\end{itemize}
	\item statische Kopplung
	\begin{itemize}
		\item D/K-Interface $v_K = \gamma_{DK}(x_D(k))$ (reelle Größen, binär)
		\item K/D-Interface $v_D = \gamma_{KD}(x_K(t))$ (binär)
	\end{itemize}
\end{itemize}

gesteuertes PN: \beiblatt{6-2}

\subsection{Simulation, Analyse und Steuerung hybrider Systeme}
Hier: Einsatz des NZMs
Ausführung im Detail: (s. at-Aufsatz)

\begin{enumerate}
	\item \underline{Simulation}
	%TODO Abb.
	\item \underline{Analyse} \label{enum:analysehybrid}
	Zentrale Fragestellung: Erreichbarkeit (von gewünschten/verbotenen Zuständen)
	
	Hierzu: bidirektionale Erreichbarkeitsanalyse 
	
	Beispiel
	%TODO Abb 
	\item \underline{Steuerung}
	Auf Basis von \ref{enum:analysehybrid}: Trajektorienplanung (= Ermittlung der gewünschten Folge von Steuereingriffen)
	
	Struktur der Steuerung: \beiblatt{6-3}
\end{enumerate}

\subsection{Beispiel: Zweitanksystem}
(s. at-Aufsatz)


