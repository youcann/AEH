\subsubsection{Netzdynamik} \label{sec:netzdynamik}
$\hat{=}$ Markenfluss im PN, bewirkt durch \textit{Schaltvorgänge} aktivierter Transitionen (s. \textit{BB AEH 2-8})

\properparagraph{Beispiel}
%TODO Abbn.

\properparagraph{Konfliktlösung (mittels Zusatzwissen)}
\begin{itemize}
	\item zusätzliche bool'sche Bedingung an den Transitionen
	\item Kantenzeitbewertung (siehe später)
	\item alternierendes Schalten
	\item Zufallsauswahl
\end{itemize}

\properparagraph{Demo: DESSKA}
%TODO Abb.

\subsubsection{Netzklassen}
\underline{Netzklassen:} Netz mit bestimmten (beschränkten) Eigenschaften zur Abbildung der jeweils charakteristischen dynamischen Abläufe (Beschränkung vorteilhaft für die Analyse/Synthese).

Klassifizierung nach:
\begin{itemize}
	\item Struktureigenschaften
	\item Stellen-/Kanteneigenschaften 
	\smash{\raisebox{.5\dimexpr1\baselineskip+4\itemsep+2\parskip}{$\left.\rule{0pt}{.5\dimexpr2\baselineskip+3\itemsep+3\parskip}\right\}\text{(s. \textit{BB AEH 2-9 und AEH 2-10})}$}} 
\end{itemize}

\properparagraph{Erläuterungen Struktureigenschaften}
\begin{itemize}
	\item \textbf{\underline{Reines PN}}
	
	Schleifen verboten
	%TODO Abb.
	
	\item \textbf{\underline{Zustandsmaschine (ZM)}}
	
	\underline{Beispiel:}
	%TODO Abb.
	
	Kennzeichen:
	\begin{itemize}
		\item Abbildung alternativer Abläufe
		\item Markenzahl in ZM konstant
		\item Markenanzahl:
		\begin{itemize}
			\item mehrere Marken: Nebenläufigkeit darstellbar (aber: $\forall s_i: K(s_i)>1$ erforderlich)
			\item genau eine Marke: ZM $\hat{=}$ Automatengraph
		\end{itemize}
	\end{itemize}

	\item \underline{Synchronisationsgraph (SG)} (s. \textit{BB AEH 2-10})
	
	\underline{Beispiel:}
	%TODO Abb.
	
	Kennzeichen:
	\begin{itemize}
		\item Modellierung von Nebenläufigkeit
		\item keine Alternativen abbildbar (dadurch konfliktfrei)
		\item Markenzahl nicht konstant
	\end{itemize}

	\item \underline{Free-Choice-Netze (FC-Netz)} (s. \textit{BB AEH 2-10})
	
	Kennzeichen:
	\begin{itemize}
		\item Kombination von ZM- und SG-Strukturen \newline ($\Rightarrow$ ZM $\subset$ FC-Netz, SG $\subset$ FC-Netz)
		\item bei Alternativen ist nur der Ausgangszustand (Konfliktstelle) betroffen, kein Einfluss anderer Zustände (Stellen)
	\end{itemize}
\end{itemize}

\properparagraph{Anmerkungen}
\begin{enumerate}
	\item Erweiterung des FC-Netz: \underline{Extended Free-Choice-Netz (EFC-Netz)}
	
	gegenseitig bedingte Alternativen zulässig
	%TODO Abb.
	
	\item allgemeinstes PN: \underline{Stellen/Transitionen-Netz (S/T-Netz)}
	
	Alle Struktur- und Knoten/Kanteneigenschaften erlaubt (demzufolge alle übrigen Netzklassen).
\end{enumerate}

\properparagraph{Beispiele}
%TODO Abb.


\includepdf[pages={18 - 20}]{material/AEH_2019.pdf}